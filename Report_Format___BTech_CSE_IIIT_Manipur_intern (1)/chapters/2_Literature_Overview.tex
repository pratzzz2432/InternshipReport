\chapter{Literature Overview}

\section{Introduction to the Literature Survey}
This literature survey provides a comprehensive review of the technologies, methodologies, and business contexts relevant to the internship project at wrkin.app. The survey covers modern web development frameworks, database and infrastructure technologies, security and authentication mechanisms, software engineering practices, and the specific business domain of distributor onboarding. The aim is to ground the internship work in existing academic and industry literature, highlighting the state-of-the-art and identifying areas for innovation.

\section{Modern Web Development Frameworks}

\subsection{React Ecosystem}
The React ecosystem, maintained by Facebook, is a cornerstone of modern frontend development. Its component-based architecture allows for the creation of reusable UI components, which is particularly beneficial for large-scale applications. Key aspects of the React ecosystem include:
\begin{itemize}
    \item \textbf{React.js:} A JavaScript library for building user interfaces.
    \item \textbf{Next.js:} A framework for building server-rendered React applications.
    \item \textbf{State Management:} Libraries like Redux and MobX for managing application state.
\end{itemize}

\subsection{Node.js and NestJS}
Node.js is a JavaScript runtime that allows for the execution of JavaScript on the server-side. NestJS is a progressive Node.js framework for building efficient, reliable, and scalable server-side applications. It provides a level of abstraction above common Node.js frameworks and brings structure to Node.js applications.

\section{Database and Infrastructure Technologies}

\subsection{MongoDB in Enterprise Applications}
MongoDB is a popular NoSQL database that uses a document-oriented data model. It is well-suited for applications that require high performance, high availability, and easy scalability. Its flexible schema is particularly useful in agile development environments where requirements may change frequently.

\subsection{Containerization and Cloud Infrastructure}
Containerization technologies like Docker have revolutionized the way applications are deployed and managed. They allow for the packaging of an application with all of its dependencies into a standardized unit for software development. Cloud infrastructure providers like AWS, Google Cloud, and Azure provide the tools and services to deploy and scale containerized applications.

\section{Security and Authentication}

\subsection{Passwordless Authentication}
Passwordless authentication is a method of verifying a user's identity without using a password. This can be done through various methods, such as sending a one-time code to the user's email or phone, or using biometric authentication. Passwordless authentication can improve security and user experience by eliminating the need for users to remember complex passwords.

\subsection{Generative AI Integration}
Generative AI is a type of artificial intelligence that can create new content, such as text, images, or code. It has the potential to revolutionize many industries, including software development. In the context of this internship, generative AI could be used to automate tasks, generate code, or create documentation.

\section{Software Engineering Practices}

\subsection{Code Quality and Refactoring}
Code quality is a measure of how well-written and maintainable a codebase is. Refactoring is the process of restructuring existing computer code—changing the factoring—without changing its external behavior. Refactoring is intended to improve the design, structure, and/or implementation of the software, while preserving its functionality.

\subsection{API Design and Documentation}
An Application Programming Interface (API) is a set of rules that allows different software applications to communicate with each other. Good API design is crucial for building scalable and maintainable applications. API documentation is also essential for ensuring that the API is easy to use and understand.

\section{Business Domain Literature}

\subsection{Distributor Onboarding Processes}
Distributor onboarding is the process of bringing a new distributor into a company's network. This process can be complex and time-consuming, and it is often a major pain point for both distributors and companies. There is a growing body of literature on how to improve the distributor onboarding process, with a focus on using technology to automate tasks and improve communication.

\subsection{Enterprise Software Adoption}
Enterprise software adoption is the process by which a new software system is integrated into the daily work of individuals and groups within an organization. The success of any new enterprise software system depends on its adoption by the intended users. There is a large body of research on the factors that influence enterprise software adoption, including user training, management support, and the perceived usefulness of the system.

\section{Startup Engineering Contexts}

\subsection{Engineering Practices in Startups}
Engineering practices in startups are often different from those in larger, more established companies. Startups are typically characterized by a fast-paced environment, limited resources, and a focus on rapid growth. This can lead to a more agile and experimental approach to software development.

\subsection{Internship Experiences in Software Engineering}
Internships are an important part of the software engineering curriculum. They provide students with the opportunity to apply their academic knowledge in a real-world setting and to gain valuable experience in the software industry. There is a growing body of literature on the benefits of internships for both students and employers.

\section{Synthesis and Research Gaps}
This literature survey has identified a number of key technologies, methodologies, and business contexts relevant to the internship project. However, there are also a number of research gaps. For example, there is a need for more research on the use of generative AI in software development, and on the specific challenges of distributor onboarding in the Indian market.

\section{Conclusion}
This literature survey has provided a comprehensive overview of the state-of-the-art in the technologies, methodologies, and business contexts relevant to the internship project. The survey has also identified a number of research gaps that could be addressed in future work.
