\section{Overall System Architecture}
The wrkin.app platform is built on a modern, full-stack architecture. The frontend is a mobile application developed using Flutter, while the backend is a RESTful API built with Django. The database is PostgreSQL, with Redis used for caching. The system is designed to be scalable and modular, with a microservices-oriented structure to prepare for future growth.
\newline
[INSERT FIGURE: System Architecture Diagram]

\section{Frontend Development - Flutter Framework}
The mobile application is developed using Flutter 3.x and Dart 3.x with null safety. The BLoC (Business Logic Component) pattern is used for state management. The UI is built with custom widgets to create a unique and intuitive user experience. Key features implemented include:
\begin{itemize}
    \item Real-time chat with WebSocket integration.
    \item Push notifications using Firebase Cloud Messaging.
    \item File management for multi-format uploads and downloads.
    \item Camera integration for document scanning.
    \item Biometric authentication for enhanced security.
\end{itemize}

\section{Backend Development - Django Framework}
The backend is developed using Django 4.x and Python 3.11+. The Django REST Framework is used to create a RESTful API. Key features of the backend include:
\begin{itemize}
    \item JWT-based authentication with refresh tokens.
    \item WebSocket support for real-time features via Django Channels.
    \item Cloud storage integration with AWS S3 for file management.
    \item Asynchronous task processing with Celery and Redis.
    \item Security features such as rate limiting and CORS protection.
\end{itemize}

\section{Database Design and Management}
The database schema is designed using a normalized structure with performance indexes. PostgreSQL is the primary database. The design includes complex many-to-many relationships to model users, teams, projects, and conversations. Database migrations are managed using Django's built-in migration system.
\newline
[INSERT FIGURE: Database Schema Diagram]

\section{Development Environment and Tools}
The development environment consists of Android Studio and Visual Studio Code for coding, Figma for UI/UX design, and Postman for API testing. Git is used for version control, with a feature branch workflow and and mandatory code reviews. The CI/CD pipeline includes automated testing to ensure code quality.